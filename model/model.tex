\documentclass{article}
\usepackage{amsmath}
\usepackage{amssymb}

\begin{document}

\title{Modelo de Optimización Lineal Entera Mixta (MILP)}
\author{}
\date{}
\maketitle

\section{Definición del Problema}

Se desea minimizar el costo total de construcción y operación de plantas de generación eléctrica mientras se satisface la demanda energética y se cumplen las restricciones ambientales.

\subsection{Datos del Problema}

\begin{itemize}
    \item Tipos de Plantas:
    \begin{itemize}
        \item Térmica (T)
        \item Hidroeléctrica (H)
        \item Renovable (R)
    \end{itemize}
    \item Costos:
    \begin{itemize}
        \item Costo de construcción: 
        \begin{align*}
            &\text{T: } 1,000,000 \text{ USD} \\
            &\text{H: } 2,000,000 \text{ USD} \\
            &\text{R: } 800,000 \text{ USD}
        \end{align*}
        \item Costo de generación:
        \begin{align*}
            &\text{T: } 50 \text{ USD/MW} \\
            &\text{H: } 30 \text{ USD/MW} \\
            &\text{R: } 20 \text{ USD/MW}
        \end{align*}
    \end{itemize}
    \item Emisiones de CO$_2$:
    \begin{align*}
        &\text{T: } 0.8 \text{ ton/MW} \\
        &\text{H, R: } 0 \text{ ton/MW}
    \end{align*}
    \item Generación mínima y máxima por planta:
    \begin{align*}
        &\text{T: } 50 \leq g_{Ti} \leq 200 \\
        &\text{H: } 100 \leq g_{Hj} \leq 300 \\
        &\text{R: } 20 \leq g_{Rk} \leq 150
    \end{align*}
    \item Restricciones:
    \begin{itemize}
        \item Demanda total: 1000 MW.
        \item Límite de CO$_2$: 500 toneladas.
        \item Máximo de plantas por tipo: 10.
    \end{itemize}
\end{itemize}


\subsection{Variables}

Variables enteras:
\begin{align*}
    x_T, x_H, x_R \in \mathbb{Z}^+ \quad &\text{Número de plantas a construir de cada tipo (entre 0 y 10).}
\end{align*}

Variables continuas:
\begin{align*}
    g_{Ti}, g_{Hj}, g_{Rk} \geq 0 \quad &\text{Generación de cada planta térmica, hidroeléctrica y renovable.}
\end{align*}

\subsection{Función Objetivo}

Minimizar el costo total de construcción y generación:
\begin{align*}
    \min Z &= \sum_{i=1}^{x_T} (1,000,000) + \sum_{j=1}^{x_H} (2,000,000) + \sum_{k=1}^{x_R} (800,000) \\
    &+ \sum_{i=1}^{x_T} (50 g_{Ti}) + \sum_{j=1}^{x_H} (30 g_{Hj}) + \sum_{k=1}^{x_R} (20 g_{Rk}).
\end{align*}

\subsection{Restricciones}

\textbf{1. Satisfacción de la demanda energética:}
\begin{align*}
    \sum_{i=1}^{x_T} g_{Ti} + \sum_{j=1}^{x_H} g_{Hj} + \sum_{k=1}^{x_R} g_{Rk} \geq 1000.
\end{align*}

\textbf{2. Límites de generación por planta:}
\begin{align*}
    50 \leq g_{Ti} \leq 200, \quad &\forall i = 1, \dots, x_T, \\
    100 \leq g_{Hj} \leq 300, \quad &\forall j = 1, \dots, x_H, \\
    20 \leq g_{Rk} \leq 150, \quad &\forall k = 1, \dots, x_R.
\end{align*}

\textbf{3. Restricción de emisiones de CO$_2$:}
\begin{align*}
    \sum_{i=1}^{x_T} 0.8 g_{Ti} \leq 500.
\end{align*}

\textbf{4. Límite en la cantidad de plantas:}
\begin{align*}
    0 \leq x_T, x_H, x_R \leq 10, \quad x_T, x_H, x_R \in \mathbb{Z}^+.
\end{align*}

\end{document}
